\documentclass[11pt]{article}
\usepackage[T1]{fontenc}
\usepackage[german]{babel}
\usepackage[utf8]{inputenc}

%% margins
\usepackage{geometry}
\geometry{
  left=2.5cm,
  right=2.5cm,
  top=2.5cm,
  bottom=2.5cm
}

%% ams/thm
\usepackage{amsmath, amssymb, amsthm, mathtools, cancel, graphicx, wrapfig}

%%\begin{thm config/technicalities}
\theoremstyle{plain}
\newtheorem{thm}{Satz}[section]
\newtheorem{lem}[thm]{Lemma}
\newtheorem*{prop}{Proposition}
\newtheorem*{cor}{Korollar}
\newtheorem*{beh}{Behauptung}

\theoremstyle{definition}
\newtheorem{defn}[thm]{Definition}
\newtheorem{exmp}[thm]{Beispiel}

\theoremstyle{remark}
\newtheorem*{rem}{Bemerkung}

\usepackage{thmtools}

\declaretheoremstyle[%
  spaceabove=1.5ex,%
  spacebelow=6pt,%
  headfont=\normalfont\itshape,%
  postheadspace=1em,%
  qed=\qedsymbol%
]{mystyle} 
\declaretheorem[name={Beweis},style=mystyle,unnumbered,
]{prf}

\renewcommand{\proof}{\prf}

%% paragraph spacing/indentation
\usepackage{parskip}
\usepackage{setspace}
\setlength{\parindent}{0cm}

\makeatletter
\def\thm@space@setup{%
  \thm@preskip=\parskip \thm@postskip=0pt
}
\makeatother
%\end{}

%% plotting
%\usepackage{pgfplots}
%\usepackage{tikz}

\usepackage{enumitem}
%\setitemize{noitemsep,topsep=0pt,parsep=0pt,partopsep=0pt}

%% lightning
\usepackage{marvosym}

%% widecheck 
%\usepackage{mathabx}

%% sans-serif font
%\renewcommand{\familydefault}{\sfdefault}

% shortcuts
\newcommand{\N}{\mathbb{N}}
\newcommand{\Z}{\mathbb{Z}}
\newcommand{\Q}{\mathbb{Q}}
\newcommand{\R}{\mathbb{R}}
\newcommand{\C}{\mathbb{C}}
\newcommand{\K}{\mathbb{K}}
\newcommand{\Sn}{\mathbb{S}_n}
\renewcommand{\S}{\mathcal{S}}

\newcommand{\D}{\displaystyle}

\newcommand{\ep}{\varepsilon}
\newcommand{\ph}{\varphi}
\newcommand{\longto}{\longrightarrow}
\DeclareMathOperator{\kgV}{\mathrm{kgV}}
\DeclareMathOperator{\ggT}{\mathrm{ggT}}
\DeclareMathOperator{\ord}{\mathrm{ord}\,}
\DeclareMathOperator{\sgn}{\mathrm{sgn}\,}
\newcommand{\id}{\mathrm{id}}
\renewcommand{\div}{\operatorname{div}}
\newcommand{\rot}{\operatorname{rot}}

\DeclareMathOperator{\Fou}{\mathcal{F}\!}

% fancy head/foot
\usepackage{fancyhdr}
\fancyhead[R]{M. Böhl, A. Kanz, R. Müller, M. Nietschmann, M. Noatsch}
\fancyhead[L]{}


\begin{document}
\pagestyle{fancy}
\thispagestyle{plain}

\rule{\textwidth}{.5pt}
\begin{center}
\Huge{Theoretische Physik I - Übungsblatt 2}
\end{center}

\rule{\textwidth}{.5pt}
\text{} \hfill M. Böhl, A. Kanz, R. Müller, M. Nietschmann, M. Noatsch


%%%%%%%%%%%%%%%%%%%%%%%%%%%%%%%
%%							 %%
%%   !!!Hier geht's los!!!   %%
%%							 %%
%%%%%%%%%%%%%%%%%%%%%%%%%%%%%%%


\section{Flussintegral bei radialer Strömung}


\section{Kurvenintegral bei Scherströmung}
Gegeben ist das Vektorfeld $v(r) = (0, x, 0)$ für $r = (x,y,z) \in \R^3$. Seien nun $k \in (0, \infty)$ und $Q(k) = \{ (x,y,0) \in \R^3 \mid \max \{ |x|,|y| \} \leq \frac{k}{2} \}$ das Quadrat mit Seitenlänge $k$ und dem Koordinatenursprung als Mittelpunkt. Um den den Rand des Quadrats zu parametrisieren, definieren wir zunächst $\gamma_i : [0,1] \longto \R^3$ für $i \in \{1,\dots, 4\}$ durch
\[ \gamma_1 (t) = (1, 2t-1, 0), \quad \gamma_2 (t) = (1-2t, 1, 0), \quad \gamma_3 (t) = (-1, 1-2t, 0), \quad \gamma_4 (t) = (2t-1, -1, 0). \]

Dann parametrisiert
\[ \gamma_k: [0,4] \longto \R^3,\; t \longmapsto \begin{cases}
\tfrac{k}{2}\gamma_1(t) &, t \in [0,1)\\
\tfrac{k}{2}\gamma_2(t-1) &, t \in [1,2)\\
\tfrac{k}{2}\gamma_3(t-2) &, t \in [2,3)\\
\tfrac{k}{2}\gamma_4(t-3) &, t \in [3,4]
\end{cases} \]

den Rand des Quadrates $Q(k)$. Nun können wir das Kurvenintegral berechnen:
\begin{align*}
\oint_{\gamma_k} v(r)\cdot dr &= \int_0^4 v(\gamma_k(t)) \cdot \dot \gamma_k (t) dt \\
&= \frac{k^2}{4} \left( \int_0^1 (0,1,0) \cdot (0,2,0) dt + \int_0^1 (0, 1-2t, 0) \cdot (-2,0,0) dt \right.\\
&\quad \left. + \int_0^1 (0,-1,0) \cdot (0,-2,0) dt + \int_0^1 (0, 2t-1, 0) \cdot (2,0,0) dt \right)\\
&= 2\cdot\frac{k^2}{4} \int_0^1 2 dt = k^2.
\end{align*}

Die Rotation berechnet sich als $\rot v(r) = (0,0,1)$.


\section{Kurvenintegral eines azimutalen Geschwindigkeitsfeldes}
Gegeben ist das Vektorfelt $v(r) = (-y, x, 0)$ für $r = (x,y,z) \in \R^3$. Sei $R \in (0, \infty)$ und $K(R)$ der konzentrische Kreis mit Radius $R$ in der Ebene $z = 0$. Der Rand von $K(R)$ wird von der Kurve
\[ \gamma_R : [0, 2\pi) \longto \R^3, \; \ph \longmapsto R (\cos\ph, \sin\ph, 0) \]
``gegen den Urzeigersinn umfahren''. Es gilt
\[ \oint_{\gamma_R} v(r) \cdot dr = R^2 \int_0^{2\pi} (-\sin\ph, \cos\ph, 0) \cdot (-\sin\ph, \cos\ph, 0) d\ph = R^2 \int_0^{2\pi} \sin^2 \ph + \cos^2 \ph d\ph = 2\pi R^2 \]
und $\rot v(r) = (0,0,2)$.


\section{Linienintegral}
Gegeben ist das Vektorfeld $f(x,y,z) = xyz(1,1,1)$. Wir berechnen nun das Integral $\int_\gamma f(r) dr$ für verschiedene Kurven $\gamma$.
\begin{enumerate}[label = \alph*)]
\item Definiere zunächst $\gamma_i : [0,1] \longto \R^3$ für $i \in \{1,\dots, 3\}$ durch
\[ \gamma_1(t) = (t,0,0), \qquad \gamma_2(t) = (1,t,0), \qquad \gamma_3(t) = (1,1,t). \]
Sei nun 
\[ \gamma : [0,3) \longto \R^3, \; t \longmapsto \gamma_i(t) \text{ falls } t \in [i-1, i) \text{ für } i \in \{1,2,3\}. \]
Dann gilt
\begin{align*}
\int_\gamma f(r) \cdot dr &= \int_0^1 (0,0,0) \cdot \dot\gamma_1(t) dt + \int_0^1 (0,0,0) \cdot \dot\gamma_2(t) dt + \int_0^1 t(1,1,1) \cdot (0,0,1) dt \\
&= \int_0^1 t dt = \frac{1}{2} t \Big\vert_0^1 = \frac{3}{4}.
\end{align*}

\item Sei $\gamma : [0,1] \longto \R^3, \; t \longmapsto (t,t,t)$. Dann
\[\int_\gamma f(r) \cdot dr = \int_0^1 t^3(1,1,1)\cdot (1,1,1) dt = \int_0^1 3t^3 dt = \frac{3}{4} t^4 \Big\vert_0^1 = \frac{3}{4}. \]

\item Sei $\gamma : [0,1] \longto \R^3, \; t \longmapsto (t^2, t^3, t^4)$. Dann
\[ \int_\gamma f(r) \cdot dr = \int_0^1 t^9(1,1,1)\cdot (2t, 3t^2, 4t^3) dt = \int_0^1 2t^{10} + 3t^{11} + 4t^{12} dt = \frac{2}{11} + \frac{3}{12} + \frac{4}{13} = \frac{423}{572}. \]
\end{enumerate}

Außerdem gilt $\rot f(x,y,z) = \begin{pmatrix} xz - xy \\ xy - yz \\ yz - xz \end{pmatrix}$.


\end{document}