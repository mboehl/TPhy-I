\documentclass[11pt]{article}
\usepackage[T1]{fontenc}
\usepackage[german]{babel}
\usepackage[utf8]{inputenc}

%% margins
\usepackage{geometry}
\geometry{
  left=2.5cm,
  right=2.5cm,
  top=2.5cm,
  bottom=2.5cm
}

%% ams/thm
\usepackage{amsmath, amssymb, amsthm, mathtools, cancel, graphicx, wrapfig}

%%\begin{thm config/technicalities}
\theoremstyle{plain}
\newtheorem{thm}{Satz}[section]
\newtheorem{lem}[thm]{Lemma}
\newtheorem*{prop}{Proposition}
\newtheorem*{cor}{Korollar}
\newtheorem*{beh}{Behauptung}

\theoremstyle{definition}
\newtheorem{defn}[thm]{Definition}
\newtheorem{exmp}[thm]{Beispiel}

\theoremstyle{remark}
\newtheorem*{rem}{Bemerkung}

\usepackage{thmtools}

\declaretheoremstyle[%
  spaceabove=1.5ex,%
  spacebelow=6pt,%
  headfont=\normalfont\itshape,%
  postheadspace=1em,%
  qed=\qedsymbol%
]{mystyle} 
\declaretheorem[name={Beweis},style=mystyle,unnumbered,
]{prf}

\renewcommand{\proof}{\prf}

%% paragraph spacing/indentation
\usepackage{parskip}
\usepackage{setspace}
\setlength{\parindent}{0cm}

\makeatletter
\def\thm@space@setup{%
  \thm@preskip=\parskip \thm@postskip=0pt
}
\makeatother
%\end{}

%% plotting
\usepackage{pgfplots}
\usepackage{tikz}

\usepackage{enumitem}
%\setitemize{noitemsep,topsep=0pt,parsep=0pt,partopsep=0pt}

%% lightning
\usepackage{marvosym}

%% widecheck 
%\usepackage{mathabx}

%% sans-serif font
%\renewcommand{\familydefault}{\sfdefault}

% shortcuts
\newcommand{\N}{\mathbb{N}}
\newcommand{\Z}{\mathbb{Z}}
\newcommand{\Q}{\mathbb{Q}}
\newcommand{\R}{\mathbb{R}}
\newcommand{\C}{\mathbb{C}}
\newcommand{\K}{\mathbb{K}}
\newcommand{\Sn}{\mathbb{S}_n}
\renewcommand{\S}{\mathcal{S}}

\newcommand{\D}{\displaystyle}
\renewcommand{\d}{\,\mathrm{d}}

\newcommand{\ep}{\varepsilon}
\newcommand{\ph}{\varphi}
\renewcommand{\th}{\vartheta}
\newcommand{\longto}{\longrightarrow}
\DeclareMathOperator{\kgV}{\mathrm{kgV}}
\DeclareMathOperator{\ggT}{\mathrm{ggT}}
\DeclareMathOperator{\ord}{\mathrm{ord}\,}
\DeclareMathOperator{\sgn}{\mathrm{sgn}\,}
\newcommand{\id}{\mathrm{id}}
\renewcommand{\div}{\operatorname{div}}
\newcommand{\rot}{\operatorname{rot}}

\DeclareMathOperator{\Fou}{\mathcal{F}\!}

% fancy head/foot
\usepackage{fancyhdr}
\fancyhead[R]{M. Böhl, A. Kanz, R. Müller, M. Nietschmann}
\fancyhead[L]{}


\begin{document}
\pagestyle{fancy}
\thispagestyle{plain}

\rule{\textwidth}{.5pt}
\begin{center}
\Huge{Theoretische Physik I - Übungsblatt 2}
\end{center}

\rule{\textwidth}{.5pt}
\text{} \hfill M. Böhl, A. Kanz, R. Müller, M. Nietschmann


%%%%%%%%%%%%%%%%%%%%%%%%%%%%%%%
%%									  %%
%%   !!!Hier geht's los!!!   %%
%%									  %%
%%%%%%%%%%%%%%%%%%%%%%%%%%%%%%%


\section{Flussintegral bei radialer Strömung}
Gegeben ist das Vektorfeld $v(\vec r) = \frac{c}{r^3} \vec r$ für $\vec r \in \R^3$ mit $r = ||\vec r ||$ und $c \in \R$.

Um den Fluss durch die Konzentrische Kugel $K$ mit Radius $R$ zu berechnen, parametrisieren wir $\partial K$ wie üblich durch 
\[ f: [0, \pi) \times [0, 2\pi) \longrightarrow \R^3, \; (\th, \ph) \longmapsto R \begin{pmatrix}
\sin\th \cos\ph \\ \sin\th \sin \ph \\ \cos\th
\end{pmatrix}. \]
Dann folgt
\[ f_\th = \partial_\th f(\th, \ph) = R \begin{pmatrix}
\cos\th \cos\ph \\ \cos\th \sin\ph \\ -\sin\th
\end{pmatrix}, 
\qquad f_\ph = \partial_\ph f(\th, \ph) = R \begin{pmatrix}
-\sin\th \sin\ph \\ \sin\th \cos\ph \\ 0
\end{pmatrix} \]
und damit
\[ f_\th \times f_\ph = R^2 \begin{pmatrix}
\sin^2 \th \cos\ph \\ \sin^2 \th \sin\ph \\ \cos\th \sin\th(\sin^2 \ph + \cos^2 \ph)
\end{pmatrix}
= R^2 \sin\th \begin{pmatrix}
\sin\th \cos\ph \\ \sin\th \sin \ph \\ \cos\th
\end{pmatrix}. \]
Nun zur Berechnung des Flusses durch $K$:
\begin{align*}
\int_K \div \vec v \d V &= \int_{\partial K} \vec v \cdot \d \vec a \\
&= \int_{0}^{2\pi} \int_{0}^{\pi} \vec v(f(\th, \ph)) \cdot (f_\th \times f_\ph) \d \th \d \phi\\
&= \int_{0}^{2\pi} \int_{0}^{\pi} \frac{c}{R^2} \begin{pmatrix}
\sin\th \cos\ph \\ \sin\th \sin \ph \\ \cos\th
\end{pmatrix} \cdot \begin{pmatrix}
\sin\th \cos\ph \\ \sin\th \sin \ph \\ \cos\th
\end{pmatrix} R^2 \sin\th \d\th \d\ph \\
&= c \int_{0}^{2\pi} \int_{0}^{\pi} \sin\th (\sin^2 \th \cos^2 \ph +  \sin^2 \th \sin^2 \ph + \cos^2 \th) \d\th \d\ph \\
&= c \int_{0}^{2\pi} \int_{0}^{\pi} \sin\th \d\th \d\ph \\
&= 2\pi c (-\cos\th\vert_0^\pi) = 0.
\end{align*}

\begin{center}
\begin{tikzpicture}
\begin{axis}[
	title = {Vektorfeld $v(\vec r) = \frac{c}{r^3} \vec r$},
	domain = -3:3,
	xmin = -2, xmax = 2,
	ymin = -2, ymax = 2,
	axis equal,
	%xlabel = $x$, ylabel = $y$,
	view = {0}{90},
]
	\addplot3[black,
		quiver = {u={x/(x*x+y*y)}, v={y/(x*x+y*y)}, scale arrows = 0.2},
		-stealth, samples=20] {0};
		
\end{axis}
\end{tikzpicture}
\end{center}

\section{Kurvenintegral bei Scherströmung}
Gegeben ist das Vektorfeld $\vec v(\vec r) = (0, x, 0)$ für $\vec r = (x,y,z) \in \R^3$. Seien nun $k \in (0, \infty)$ und $Q(k) = \{ (x,y,0) \in \R^3 \mid \max \{ |x|,|y| \} \leq \frac{k}{2} \}$ das Quadrat mit Seitenlänge $k$ und dem Koordinatenursprung als Mittelpunkt. Um den den Rand des Quadrats zu parametrisieren, definieren wir zunächst $\gamma_i : [0,1] \longto \R^3$ für $i \in \{1,\dots, 4\}$ durch
\[ \gamma_1 (t) = (1, 2t-1, 0), \quad \gamma_2 (t) = (1-2t, 1, 0), \quad \gamma_3 (t) = (-1, 1-2t, 0), \quad \gamma_4 (t) = (2t-1, -1, 0). \]

Dann parametrisiert
\[ \gamma_k: [0,4] \longto \R^3,\; t \longmapsto \begin{cases}
\tfrac{k}{2}\gamma_1(t) &, t \in [0,1)\\
\tfrac{k}{2}\gamma_2(t-1) &, t \in [1,2)\\
\tfrac{k}{2}\gamma_3(t-2) &, t \in [2,3)\\
\tfrac{k}{2}\gamma_4(t-3) &, t \in [3,4]
\end{cases} \]

den Rand des Quadrates $Q(k)$. Nun können wir das Kurvenintegral berechnen:
\begin{align*}
\oint_{\gamma_k} \vec v(\vec r)\cdot \d\vec r &= \int_0^4 \vec v(\gamma_k(t)) \cdot \dot \gamma_k (t) \d t \\
&= \frac{k^2}{4} \left( \int_0^1 (0,1,0) \cdot (0,2,0) dt + \int_0^1 (0, 1-2t, 0) \cdot (-2,0,0) \d t \right.\\
&\quad \left. + \int_0^1 (0,-1,0) \cdot (0,-2,0) dt + \int_0^1 (0, 2t-1, 0) \cdot (2,0,0) \d t \right)\\
&= 2\cdot\frac{k^2}{4} \int_0^1 2 \d t = k^2.
\end{align*}

\begin{center}
\begin{tikzpicture}
\begin{axis}[
	title={Scherströmung},
	domain=-2:2,
	xmin=-1.8, xmax=1.8,
	ymax=2, ymin=-2,
	%axis equal,
	xtick={-1,0,1},
	view={0}{90}
]	
	\addplot3[black,
		quiver={
			u={0},
			v={x},
			scale arrows=.4,
		},
		-stealth,samples=8]
			{0};

\end{axis}
\end{tikzpicture}
\end{center}

%\begin{tikzpicture}
%\begin{axis}[
%	title={Scherströmung 2?}
%]
%%	\addplot[gray,
%%		quiver={u=0, v=x},
%%		-stealth] {-x};
%	
%	\addplot[black,
%		quiver={u=0, v=x},
%		-stealth] {0};
%
%\end{axis}		
%\end{tikzpicture}


\section{Kurvenintegral eines azimutalen Geschwindigkeitsfeldes}
Gegeben ist das Vektorfeld $\vec v(\vec r) = (-y, x, 0)$ für $\vec r = (x,y,z) \in \R^3$. Sei $R \in (0, \infty)$ und $K(R)$ der konzentrische Kreis mit Radius $R$ in der Ebene $z = 0$. Der Rand von $K(R)$ wird von der Kurve
\[ \gamma_R : [0, 2\pi) \longto \R^3, \; \ph \longmapsto R (\cos\ph, \sin\ph, 0) \]
``gegen den Uhrzeigersinn umfahren''. Es gilt
\[ \oint_{\gamma_R} \vec v(\vec r) \cdot \d\vec r = R^2 \int_0^{2\pi} (-\sin\ph, \cos\ph, 0) \cdot (-\sin\ph, \cos\ph, 0) \d\ph = R^2 \int_0^{2\pi} \sin^2 \ph + \cos^2 \ph \d\ph = 2\pi R^2. \]

\begin{center}
\begin{tikzpicture}
\begin{axis}[
	title={Azimutales Geschwindigkeitsfeld},
	view={0}{90},
	axis equal,
	domain=-2:2,
	ymin=-1, ymax=1,
	xmin=-1, xmax=1,
	%xlabel=$x$, ylabel=$y$,
	]
	\addplot3[black,
		quiver={u={-y}, v={x}, scale arrows=0.2},
		-stealth,
		samples=20] {0};
		
%	\addplot3[red, samples=50, domain=0:2*pi]
%		({cos(deg(x))}, {sin(deg(x))}, 0);

\end{axis}
\end{tikzpicture}
\end{center}

\newpage
\section{Linienintegral}
Gegeben ist das Vektorfeld $\vec f(x,y,z) = xyz(1,1,1)$. Wir berechnen nun das Integral $\int_\gamma \vec f(\vec r) \cdot d\vec r$ für verschiedene Kurven $\gamma$.
\begin{enumerate}[label = \alph*)]
\item Definiere zunächst $\gamma_i : [0,1] \longto \R^3$ für $i \in \{1,\dots, 3\}$ durch
\[ \gamma_1(t) = (t,0,0), \qquad \gamma_2(t) = (1,t,0), \qquad \gamma_3(t) = (1,1,t). \]
Sei nun 
\[ \gamma : [0,3) \longto \R^3, \; t \longmapsto \gamma_i(t) \text{ falls } t \in [i-1, i) \text{ für } i \in \{1,2,3\}. \]
Dann gilt
\begin{align*}
\int_\gamma \vec f(\vec r) \cdot \d\vec r &= \int_0^1 (0,0,0) \cdot \dot\gamma_1(t) \d t + \int_0^1 (0,0,0) \cdot \dot\gamma_2(t) \d t + \int_0^1 t(1,1,1) \cdot (0,0,1) \d t \\
&= \int_0^1 t \d t = \frac{1}{2} t \Big\vert_0^1 = \frac{3}{4}.
\end{align*}

\item Sei $\gamma : [0,1] \longto \R^3, \; t \longmapsto (t,t,t)$. Dann
\[\int_\gamma \vec f(\vec r) \cdot \d\vec r = \int_0^1 t^3(1,1,1)\cdot (1,1,1) \d t = \int_0^1 3t^3 \d t = \frac{3}{4} t^4 \Big\vert_0^1 = \frac{3}{4}. \]

\item Sei $\gamma : [0,1] \longto \R^3, \; t \longmapsto (t^2, t^3, t^4)$. Dann
\[ \int_\gamma \vec f(\vec r) \cdot \d\vec r = \int_0^1 t^9(1,1,1)\cdot (2t, 3t^2, 4t^3) \d t = \int_0^1 2t^{10} + 3t^{11} + 4t^{12} \d t = \frac{2}{11} + \frac{3}{12} + \frac{4}{13} = \frac{423}{572}. \]
\end{enumerate}

\section{Gaußscher Satz im $\R^2$}
Gegeben sind die Vektorfelder $\vec R = \frac{1}{\rho} \hat \rho = \nabla\ln \rho \hat \rho$ und $\vec T = \rho \hat \ph = \nabla \rho^2 \ph\hat \ph$. Sei $M \subseteq \R^2$ eine kompakte Menge mit glattem Rand. Dann gilt
\[ \int_M \div \vec R \d A = \int_M \frac{1}{\rho} \frac{\d}{\d \rho}(\rho R_\rho) + \frac{1}{\rho} \frac{\d R_\ph}{\d \ph}\d A = \int_M \frac{1}{\rho} \frac{\d}{\d \rho} 1 + 0 \d A = \int_M 0 \d A = 0. \]
und analog
\[ \int_M \div \vec T \d A = \int_M \frac{1}{\rho} \frac{\d}{\d \rho}(\rho T_\rho) + \frac{1}{\rho} \frac{\d T_\ph}{\d \ph}\d A = \int_M 0 + \frac{1}{\rho} \frac{\d }{\d \ph} \rho \d A = \int_M 0 \d A = 0. \]
Da $\vec R$ und $\vec T$ Gradientenfelder sind, verschwinden Wegintegrale über geschlossene Kurven, also
\[ \oint_{\partial M} \vec R \cdot \d\vec r = 0\]
und
\[ \oint_{\partial M} \vec T \cdot \d\vec r = 0.\]
Alles 0 also alles ok.

\section{Oberflächenintegrale}
Berechnen Sie die folgenden Oberflächenintegrale $\int_{\partial M} \vec E \cdot \d \vec a$

\begin{enumerate}[label = \alph*)]
\item $\vec E(\vec x) = \begin{pmatrix} x + y^2 + z^2 \\ \frac{x}{z} \\ 2z - \frac{y}{x} \end{pmatrix}$ und $M = B_2(0)$. Dann folgt mit dem Satz von Gauß
\[ \int_{\partial M} \vec E(\vec x) \cdot \d\vec a = \int_M \div \vec E \d V = 3 \int_M \d V = 3\cdot\frac{4}{3} \pi 2^3 = 32\pi. \]

\item $\vec E(\vec r) = \begin{pmatrix} -y \\ x \\ z^2 \end{pmatrix}$ und $M = (0,1)^3$. Dann folgt mit Satz von Gauß
\[ \int_{\partial M} \vec E(\vec x) \cdot \d\vec a = \int_M \div \vec E \d V = \int_M 2z \d V = \int_0^1 \int_0^1 \int_0^1 2z \d x \d y \d z = 1. \]

\item $\int_{\partial M} \hat z \cdot \d \vec a$ und $M = \{ \vec x \in \R^3 \mid \Vert \vec x \Vert = 1 \land z \leq 0 \}$ die Nordhalbkugel. Diese parametrisieren wir wie üblich durch 
\[ f: [0, \tfrac{\pi}{2}) \times [0, 2\pi) \longrightarrow \R^3, \; (\th, \ph) \longmapsto \begin{pmatrix}
\sin\th \cos\ph \\ \sin\th \sin \ph \\ \cos\th
\end{pmatrix}. \]
Dann
\begin{align*} \int_{\partial M} \hat z \cdot \d \vec a &= \int_0^{\frac{\pi}{2}} \int_0^{2\pi}  \hat z \cdot 
\begin{pmatrix}
\sin\th \cos\ph \\ \sin\th \sin \ph \\ \cos\th 
\end{pmatrix} 
\sin\th \d\th \d\ph \\
&= 2\pi \int_0^{\frac{\pi}{2}} \cos\th \sin\th \d\th \\
&= 2\pi \tfrac{1}{2} \sin^2 \th\big\vert_0^{\frac{\pi}{2}} = \pi.
\end{align*}
\end{enumerate}

%\begin{tikzpicture}
%\begin{axis}[
%	domain = -3:3,
%	xmin=-2, xmax=2,
%	ymin=-2, ymax=2,
%	view = {55}{60},
%]
%	\addplot3[black,
%		quiver={u={x*y*z}, v={x*y*z}, w={x*y*z}, scale arrows = 0.3},
%		-stealth, samples = 20] {1};
%\end{axis}	
%\end{tikzpicture}


\end{document}